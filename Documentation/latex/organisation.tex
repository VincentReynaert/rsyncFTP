Nous avons voulu séparer le plus possible les fonctionnalités en créant des fichiers différents: 
\begin{itemize}
\item logger.py
\item parserRsyncFTP.py
\item gestionFTP.py
\item directorySupervisor.py
\item main.py
\end{itemize}

\section{logger.py}	

Le package logger contient la gestion du logger. C'est ici qu'on s'occupe de créer le logger.

\section{parserRsyncFTP.py}

Le package parserRsyncFTP contient la fonction de définition de gestion du parser. 
C'est ici que nous définissons les paramètres que nous passons en lignes de commandes.

\section{gestionFTP.py}

Le package gestionFTP contient toutes les fonctions utiles pour gérer les actions basiques avec le serveur FTP.

\section{directorySupervisor.py}

Le package directorySupervisor correspond à la supervision des dossiers. 
Il s'agit des fonctions du tp1 que nous avons adaptées pour rsyncFTP. 
Nous n'en avons repris que le coeur de façon à garder une certaine souplesse par rapport à ce que nous avions déjà réalisé.
C'est à dire que nous l'avons amélioré notamment en supprimant les variables globales qui était dangereuses.\\
\\
Cependant, il y a toujours des cas où ce module ne fonctionne pas. Nous n'avons pas su mettre en évidence ces cas.
Nous pensons que de réaliser trop d'actions dans une courte période peut générer ces problèmes.

\section{main.py}

Le package main contient les fonctions principales. 
Nous avons chercher à le réduire à l'essentiel de façon à ce qu'il reste lisible et compréhensible au premier coup d'oeil.\\
\\
Les quelques étapes que nous réalisons sont les suivantes :
\begin{itemize}
\item Nous définissons notre parser.
\item Nous initialisons le logger.
\item Nous lançons ensuite notre boucle principale qui supervise le dossier et synchornise le répertoire situé sur le serveur FTP distant.
\end{itemize}
