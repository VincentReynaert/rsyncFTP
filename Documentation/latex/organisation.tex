Nous avons voulu séparer le plus possible les fonctionnalités en créant des fichier différents: <liste des fichiers>

\section{logger.py}	

Le package logger contient la gestion du logger.

\section{parser.py}

Le package parser contient la fonction de définition de gestion du parser. C'est ici que nous définissons les paramètres que nous passons en lignes de commandes.

\section{gestionFTP.py}

Le package gestionFTP contient toutes les fonctions utiles pour gérer les actions avec le serveur FTP.

\section{directorySupervisor.py}

Le package directorySupervisor correspond à la supervision des dossiers. Il s'agit des fonctions du tp1 que nous avons adaptées pour rsyncFTP.

\section{main.py}

Le package main contient les fonctions principales. Les quelques étapes que nous réalisons sont les suivantes :
\begin{itemize}
\item Nous définissons notre parser.
\item Nous initialisons le logger.
\item Nous lançons ensuite notre boucle principale.
\end{itemize}
